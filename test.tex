\documentclass[ebook, openany, oldfontcommands,twocolumn, 10pt]{momento}
\usepackage{lipsum}

\usepackage{pgfplots}
\usepackage{physicsplus}

\usepackage{breqn}

\title{Uno Momento}
\author{Daniel Williams}
\begin{document}

\maketitle

\tableofcontents

\lipsum

\chapter{Making Notes}
\label{cha:making-notes}

\lipsum \lipsum

\part{A part with a very very very very long name}
\label{part:part}

\lipsum
\lipsum

\section{A section of stuff.}
\label{sec:section-stuff}

\begin{fequation}[Strange Equation]
   H = \eqnote{Y^2}{Second longer} + U 
\end{fequation}

\lipsum

\subsection{More section}
\label{sec:more-section}

\lipsum[3]
\[
E = mc^2
\]
\lipsum[2]

\chapter{More Note making}
\label{cha:more-note-making}

\lipsum \lipsum \lipsum

\tinygraph[muted-blue]{(x<3 || x>7)?1:x^2}{0:10}

\begin{example}
  \lipsum
\end{example}


\begin{expl}
  %\clearevenside \switchcolumn*[] \switchcolumn[3]

  \section{Free Scalar Fields}
  \label{sec:sect-expl-envir}

  \switchcolumn*[] \switchcolumn[1]

  A scalar field associates a scalar with every point in a physical
  space. The scalar may be a number or a physical quantity, and must
  be coordinate invariant.

  \begin{illustration}
    Some examples of free scalar fields are temperature distributions
    in space, and the Higgs field.
  \end{illustration}

  Scalar field theory constitutes the simplest possible field theory,
  with a Lagrangian
  \begin{equation}
    \label{eq:15}
    \Lag = \half \partial_{\mu} \phi \partial^{\mu} \phi - \half m^2 \phi^2
  \end{equation}

  with the Euler-Lagrange equations giving a wave equation, the
  Klein-Gordon equation.


\begin{fequation}[Klein-Gordon Equation]
  \label{eq:kleingordonscalar}
  (\partial^2 + m^2) \phi = 0
\end{fequation}

\begin{derivation}
  The components of the Euler-Lagrange equations from the Lagrangian
  in equation \eqref{eq:15} are
  \begin{subequations}
    \begin{align}
      \pdv{\Lag}{\phi} &= -m^2 \phi \\
      \pdv{\Lag}{(\partial_{\mu} \phi)} &= \partial^{\mu} \phi \\
      \text{so \quad} \partial_{\mu} \qty(\pdv{\Lag}{(\partial_{\mu}
        \phi)}) &= \partial_{\mu} \partial^{\mu} \phi
    \end{align}
  \end{subequations}
\end{derivation}

which is the relativistic relation between energy and momentum---a
relativistic Schr\"odinger equation.

\begin{illustration} We can see that the Klein-Gordon equation is a
  relativistic statement of Schr\"odinger's equation by considering
  the equation's form in non-covariant notation.
  \begin{equation*}
    \label{eq:16} \qty( - \pdv[2]{t} + \nabla^2 ) \phi = m^2 \phi
  \end{equation*}
\end{illustration}

By taking a plane wave solution we have the general form of the
potential being
  \begin{dmath}
    \phi(x) = \int \frac{\dd[3]{k}}{(2 \pi)^3 2 E(\vec{k})} \left(
      a(\vec{k}) \exp(- i k \vdot x) + a^{*}(\vec{k}) \exp(i k \vdot
      x) \right)
  \end{dmath}

\begin{derivation}

  Taking a plane wave solution of equation \ref{eq:kleingordonscalar},
  \[ \phi \propto \exp(i k \vdot x) = \exp( i (k^0 t - \vec{k} \vdot
  \vec{x})) \] which requires that
  \[ (k^0)^2 - \vec{k}^2 = m^2 \] Then, this allows
  \begin{equation}
    \label{eq:17}
    k^0 = \pm \sqrt{k^2 + m^2}
  \end{equation}
  where both the positive and the negative solutions are
  valid. Letting $E(k)= + \sqrt{k^2 + m^2}$, we have a general form
  for the potential:
  \begin{dmath}
    \label{eq:18}
    \phi(x) = \int \frac{\dd[3]{k}}{(2 \pi)^3 2 E(\vec{k})} \left(
      a(-\vec{k}) e^{i\qty(E(\vec{k}) t - \vec{k} \vdot \vec{x})} +
      a^{*}(\vec{k}) e^{(i\qty(E(\vec{k}) t - \vec{k} \vdot \vec{x}))}
    \right)
  \end{dmath}
  letting $\vec{k} \to - \vec{k}$ in the first integral,
  \begin{dmath}
    \phi(x) = \int \frac{\dd[3]{k}}{(2 \pi)^3 2 E(\vec{k})} \left(
      a(\vec{k}) \exp(- i k \vdot x) + a^{*}(\vec{k}) \exp(i k \vdot
      x) \right)
  \end{dmath}
\end{derivation}

\section{The Normalisation Measure}
\label{sec:norm-meas}

The normalisation measure,
\[ \frac{\dd[3]{k}}{(2 \pi)^3 2 E(\vec{k})} \] 
while being an odd-looking quantity is the covariant choice
for normalisation. 

\begin{derivation}
  We can show that this is the covariant choice by considering a
  function $f(x)$, and then integrating it,

\begin{align*}
  \int \frac{\dd[3]{k}}{(2 \pi)^3 2 E(\vec{k})} f(k) &= \int  \frac{\dd[3]{k}}{(2 \pi)^3 2 E(\vec{k})} \dd{E^2} \delta(E^2 - \vec{k}^2 - m^2) f(k) \\
&=  \int  \frac{\dd[3]{k}}{(2 \pi)^3 2 E(\vec{k})} 2 E\dd{E} \delta(E^2 - \vec{k}^2 - m^2) f(k) \\
&=  \int  \frac{\dd[3]{k}}{(2 \pi)^3 2 E(\vec{k}) } 2 \pi\delta(E^2 - \vec{k}^2 - m^2) f(k) 
\end{align*}
where the $\delta(E^2 - \vec{k}^2)$ condition represents the
requirement for the state to be located ``on-shell''.
\end{derivation}

\section{The Energy Momentum Tensor}
\label{sec:energy-moment-tens}

The energy momentum tensor is

\newcommand{\metricten}{\ten{g}{^{\mu}^{\nu}}}

\begin{align*}
  \label{eq:19}
  \ten{T}{^{\mu}^{\nu}} &= \pdv{\Lag}{(\partial_{\mu}\phi)} \partial^{\nu} \phi - \metricten \Lag \\
&= \partial^{\mu} \phi \partial^{\nu} \phi - \metricten \qty(\half \partial_{\rho} \phi \partial^{\rho} \phi - \half m^2 \phi )
\end{align*}

Then

\begin{equation}
  \label{eq:20}
  \ten{T}{^{00}} = \partial^0 \phi \partial^0 \phi - 
     \half \qty( \partial_0 \phi \partial^0 \phi 
               - \nabla \phi \vdot \nabla \phi
               - m^2 \phi  )
\end{equation}
So the Hamiltonian is
\begin{align}
  \label{eq:21}
  H &= \int \ten{T}{^{00}} \dd[3]{x} \nonumber\\
 &= \half \int \qty( (\partial_0 \phi)^2 + (\nabla \phi)^2 + m^2\phi^2 ) \dd[3]{x}
\end{align}
This works for a classical theory, but in quantum-mechanical theories
we require $H$ to be an operator, however, we can't simply convert it
into an operator, we also need to turn the fields into operators.

\section{Second Quantisation}
\label{sec:second-quantisation}

Th second quantisation is the process of turning fields into
operators, as opposed to the approach of the first quantisation where
observables are made into operators.

In order to do this we define 
\begin{definition}[Canonically Conjugate Momentum]
  \[ \pi(x) = \pdv{\Lag}{(\partial_0 \phi)} = \partial_0 \phi(x)  \]
\end{definition}
So we can postulate that $\phi$ and $\pi$ are operators which satisfy the equal-time commutation relations
\begin{subequations}
\begin{align}
  \comm{\Op{\phi}(\vec{x}, t)}{\Op{\pi} (\vec{x}, t)} &= i \delta^3(\vec{x} - \vec{y}) \\
\comm{\Op{\phi}(\vec{x}, t)}{\Op{\phi}(\vec{y}, t)} &= \comm{\Op{\pi}(\vec{x}, t)}{\Op{\pi}(\vec{y}, t)} = 0
\end{align}
\end{subequations}

This doesn't change the Klein-Gordon equation, although it now acts
like an operator, so

\begin{equation}
  \label{eq:22}
  \Op{\phi}(\vec{x}) = \int \frac{\dd[3]{k}}{(2 \pi)^3 2 E(\vec{k}) }
                      \qty(\Op{a}(\vec{k}) e^{-ik \vdot x} + \hcon{\Op{a}}(\vec{k}) e^{i k \vdot x})
\end{equation}

where the quantities $\Op{a}$ and $\hcon{\Op{a}}$ are now operators. The corresponding equation for $\pi$ is

\begin{equation}
  \label{eq:23}
  \Op{\pi} (x) = \partial_0 \Op{\phi}(x) = \frac{i}{2} \int \frac{\dd[3]{k}}{(2 \pi)^3} \qty( - \Op{a} (\vec{k}) e^{-ik \vdot x} + \hOp{a}(\vec{k}) e^{ik \vdot x} )
\end{equation}

\section{Creation and Annihilation Operators}
\label{sec:creat-annih-oper}

\begin{subequations}
  \begin{align}
   \label{eq:28}
    \Op{a}(\vec{k}) &= \int \qty[ E(\vec{k}) \Op{\phi} + i \Op{\pi}(x) ] e^{i k \vdot x} \\
   \label{eq:29}
    \hOp{a}(\vec{k}) &= \int \qty[E(\vec{k}) \Op{\phi} - i \Op{\pi}(x) ] e^{-i k \vdot x}
  \end{align}
\end{subequations}
Which are the annihilation (\ref{eq:28}) and creation (\ref{eq:28}) operators.

\begin{derivation}
$\Op{a}$ and $\hOp{a}$ can be found through an inverse Fourier transform. 
For $\Op{\phi}$,

  \begin{align*}
   \label{eq:28}
    \Op{a}(\vec{k}) &= \int \qty[ E(\vec{k}) \Op{\phi} + i \Op{\pi}(x) ] e^{i k \vdot x} \\
   \label{eq:29}
    \hOp{a}(\vec{k}) &= \int \qty[E(\vec{k}) \Op{\phi} - i \Op{\pi}(x) ] e^{-i k \vdot x}
  \end{align*}
\begin{align*}
\int & \dd[3]{x}  \ \Op{\phi} \ e^{-ik \vdot x} \\&=
  \int \frac{\dd[3]{k'}}{(2 \pi)^3 2 E(\vec{k})} 
   \bigg( \Op{a}(\vec{k}') \int \dd[3]{x} e^{-i (k+k') \vdot x}   + \hOp{a}(\vec{k}') \int \dd[3]{x} e^{-i(k'-k) \vdot x} \bigg) 
 \\[1em] &=
 \int \frac{\dd[3]{k'}}{2 E(\vec{k})} \bigg(\Op{a}(\vec{k}') \delta^3(\vec{k} + \vec{k}') e^{-i (E(\vec{k}) - E(\vec{k}'))t}  +
 \hOp{a}(\vec{k}') \delta^3(\vec{k}' - \vec{k}) e^{i(E(\vec{k}') - E(\vec{k}))t} \bigg) 
 \\[1em] &=
\frac{1}{2 E(\vec{k})} \Big(\Op{a}(-\vec{k}) e ^{-i2E(\vec{k})t} + \hOp{a}(\vec{k}) \Big)
\end{align*}
and for $\Op{\pi}$

\begin{equation}
  \int \dd[3]{x} \Op{\pi}(x) e^{-i k \vdot x} = \frac{i}{2}
   \qty( -\Op{a}(-\vec{k}) e^{-i 2 E(\vec{k})t} + \hOp{a}(\vec{k}) )
\end{equation}

In each case using the definition of the $\delta$-function, and 
\[ 
   e^{i(k' - k) \vdot x} = e^{-i (\vec{k}' - \vec{k}) \vdot x}
                       e^{i \qty( E(\vec{k}') - E(\vec{k}) ) t}
\]

Then
\begin{subequations}
\begin{align}
\label{eq:25}
  \int \dd[3]{x} \qty[ E(\vec{k}) \Op{\phi}(x) - i \Op{\pi}(x) ]
                 e^{-i k \vdot x} &= \hOp{a}(\vec{k}) \\
\label{eq:26}
  \int \dd[3]{x} \qty[ E(\vec{k}) \Op{\phi}(x) + i \Op{\pi}(x) ]
                 e^{-i k \vdot x} &= \Op{a}(-\vec{k}) e^{-2 i E(\vec{k}) t}
\end{align}
\end{subequations}
Equation (\ref{eq:26}) is in need of some further manipulation; we
split the space and the time components, so
\begin{equation*}
  \int \qty[ E(\vec{k}) \Op{\phi}(x) + i \Op{\pi}(x)] e^{i k \vdot x} e^{-i E(\vec{k}) t} = a(-k) e^{-2i E(\vec{k} t)} 
\end{equation*}
Then, multiplying by $\exp(2 i E(\vec{k}) t)$,
\[ 
  a(-k) = \int \qty[ E(\vec{k}) \Op{\phi}(x) + i \Op{\pi}(x)] e^{i \vec{k} \vdot \vec{x}} e^{2 i E(\vec{k}) t}
\]
and replacing $\vec{k} \to - \vec{k}$,
\begin{subequations}
  \begin{align}
   \label{eq:28}
    \Op{a}(\vec{k}) &= \int \qty[ E(\vec{k}) \Op{\phi} + i \Op{\pi}(x) ] e^{i k \vdot x} \\
   \label{eq:29}
    \hOp{a}(\vec{k}) &= \int \qty[E(\vec{k}) \Op{\phi} - i \Op{\pi}(x) ] e^{-i k \vdot x}
  \end{align}
\end{subequations}

\end{derivation}



\end{expl}

\end{document}
